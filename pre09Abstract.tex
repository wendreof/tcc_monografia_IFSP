\setlength{\absparsep}{18pt} % ajusta o espaçamento dos parágrafos do resumo
\begin{resumo}[Abstract]
	
	\begin{otherlanguage*}{english}
		
		%Elemento obrigatório. É a versão do resumo em português para o idioma de divulgação internacional. Deve ser antecedido pela referência do estudo.
		
		
The evolution of mobile devices affects not only its ordinary users, but also all professionals in the field of technology, behind all the architecture of an application, be it simple or rocket science. The choice of the programming language should take into account innumerable factors, from the knowledge and experience of the involved ones, to the peculiarities regarding the ecosystem in which one aims to attain. The aim of this work is to present a qualitative and quantitative study of Java and Kotlin software development languages, with emphasis on the algorithms used in the creation of mobile applications for users of the Android mobile platform, with the objective to make the verification of which technology needs smaller amounts of characters, coding lines and size in bytes, through specific experiments like Hello World !, classes, mathematical operations and finally the elaboration of an application that makes the conversion of thermometric measurements, implemented in the languages analyzed, based on the Adapter design pattern. Given the experiments used, we obtained the result that through the Kotlin language, it is possible to reach the same results with Java, but with the least need unnecessary codes.
		
		\vspace{\onelineskip}
		 
		\textbf{Keywords}: Kotlin. Java. Development. Technology.
		
	\end{otherlanguage*}

\end{resumo} 