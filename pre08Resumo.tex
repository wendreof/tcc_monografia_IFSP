\setlength{\absparsep}{18pt} % ajusta o espaçamento dos parágrafos do resumo
\begin{resumo}
	
	%Elemento obrigatório, constituído de uma sequência de frases concisas e objetivas, fornecendo uma visão rápida e clara do conteúdo do estudo. O texto deverá conter entre 150 a 250 palavras e ser antecedido pela referência do estudo. Também, não deve conter citações e deverá ressaltar o objetivo, o método, os resultados e as conclusões. O resumo deve ser redigido em parágrafo único, seguido das palavras representativas do conteúdo do estudo, isto é, palavras-chave, em número de três a cinco, separadas entre si por ponto e finalizadas também por ponto. Usar o verbo na terceira pessoa do singular, com linguagem impessoal (pronome SE), bem como fazer uso, preferencialmente, da voz ativa.
	
	%Uma abordagem prática sobre o Kotlin, a nova linguagem de programação oficial para desenvolvimento de novos aplicativos e manutenção de aplicações legadas para a plataforma Android, abordando as suas principais características e especificidades como: tipos de tipagem, estilo de syntaxe, motivo pelo qual foi criada, sua aplicabilidade ao dia-a-dia, compatibilidade com tecnologias e técnicas já existentes e consagradas no mercado de software, estrutura arquitetural e usabilidade em ambientes de desenvolvimento integrado como IntelliJ, Netbeans, Eclipse e Android Studio e editores de texto como Visual Studio Code. Isso será possível por meio da conversão de códigos originalmente desenvolvidos em Java (uma das linguagens influenciadoras) para o Kotlin tanto de forma manual como por recursos oferecidos pelos ambientes utilizados. %(FALTAM 30 PALAVRAS)
	
	%Estudo quali-quantitativo entre as linguagens de desenvolvimento de \textit{software} \textit{Java} e \textit{Kotlin}, com ênfase nos algoritmos utilizados na criação de aplicações móveis para os usuários da plataforma móvel \textit{Android}, com o objetivo de efetuar a constatação de qual tecnologia necessita de menores quantidades de caracteres, linhas de codificação e tamanho em \textit{bytes}, por meio de experimentos específicos como programas de Olá Mundo!, classes, operações matemáticas e por fim a elaboração de uma aplicação que efetua a conversão de medidas termométricas, implementado nas linguagens analisadas, apoiando-se no padrão de projeto Adapter. Com base nos recursos 
	%utilizando-se para tal, a plataforma Android Studio e seus recursos oferecidos como formatação e endentação.
	
	A evolução dos dispositivos móveis afeta não somente os seus usuários comuns, mas também todos os profissionais do ramo de tecnologia, por traz de toda a arquitetura de uma aplicação, seja ela simples ou \textit{rocket science}. A escolha da linguagem de programação deve levar em consideração inúmeros fatores, desde conhecimento e experiência do(s) envolvido(s), até as peculiaridades referentes ao ecossistema no qual almeja-se atingir. O presente trabalho de conclusão de curso, tem o objetivo de apresentar um estudo quali-quantitativo entre as linguagens de desenvolvimento de \textit{software} \textit{Java} e \textit{Kotlin}, com ênfase nos algoritmos utilizados na criação de aplicações móveis para os usuários da plataforma móvel \textit{Android}, com o objetivo de efetuar a constatação de qual tecnologia necessita de menores quantidades de caracteres, linhas de codificação e tamanho em \textit{bytes}, por meio de experimentos específicos como programas de Olá Mundo!, classes, operações matemáticas e por fim a elaboração de uma aplicação que efetua a conversão de medidas termométricas, implementado nas linguagens analisadas, apoiando-se no padrão de projeto Adapter. Dados os experimentos utilizados, obteve-se o resultado de que por meio da linguagem Kotlin, torna-se possível chegar aos mesmos resultados com Java, porém com a menor necessidade códigos clichês e desnecessários.
	
	\vspace{\onelineskip}
	
	\textbf{Palavras-chave}: Kotlin. Java. Desenvolvimento. Tecnologia.
	
\end{resumo}